\section{Notenbildung (verbalisiert, vereinfachte Darstellung)}


{
	
\ifthenelse{\boolean{simple}}{}{\small}

\begin{enumerate}[label=\textbf{(\arabic*)}]
	\item Schriftliche und mündliche Noten sind im Verhältnis $\wsm : 1$ gewichtet.

	\ifthenelse{\boolean{simple}}{
	}{
	\item Verbesserungen können in die schriftliche Note mit einfließen.
	\begin{enumerate}[label=\textbf{(\alph*)}]
		\item Verbesserungen werden immer in Bezug zu der aktuellen Note aus Kurztests und Klassenarbeiten gewichtet.
		\item Verbesserungen fließen nur in einem Bereich von $\wth$ um halbe Noten mit in die schriftliche Leistung ein. \textit{Beispiel}: $\wth=0{,}25$; Wenn der Schnitt aus Kurztest und Klassenarbeit zwischen z.B. $2{,}25$ und $2{,}75$ liegt, dann wird jede fertige Verbesserung mit $2{,}25$ gewertet und jede Fehlende mit $2{,}75$.
		\item Fehlerhafte oder unvollständige Verbesserungen verbessern oder verschlechtern den schriftlichen Schnitt nicht.
		\item Die Gewichtung der Verbesserungen entspricht, äquivalent umgerechnet auf die jeweiligen Randbereiche des Notensystems in Bezug zu dem relativen Anteil des Einflussbereichs $\wth$, zwei Klassenarbeiten.
		\item Die Änderung der schriftlichen Note durch Verbesserungen ist auf höchstens $2 \cdot \wth$ begrenzt.
		\item Verbesserungen haben nur einen Einfluss auf die schriftliche Note, wenn sich fehlende und fertige Verbesserungen nicht in der Anzahl ausgleichen.
	\end{enumerate}
	}
	
	\item Eine Anzahl von $\nktnull$ Kurztests werden zusammen wie $\wsnull$ Klassenarbeiten gewertet. Unterschreitet die Anzahl der Kurztests $\nktnull$, wird anteilig gewertet. Bei Überschreitung werden alle Kurztests zusammen wie  $\wsnull$ Klassenarbeiten gewertet. \textit{Beispiel}:  $\nktnull = 3$; $\wsnull=1$; Werden nur zwei Kurztests geschrieben, dann werden diese zusammen soviel gewertet wie $\frac{2}{3}$ Klassenarbeiten. Werden fünf Kurztests geschrieben, werden diese zusammen soviel gewertet wie eine Klassenarbeit.
\end{enumerate}

}

