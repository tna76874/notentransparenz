\section{Kriterien für die Erteilung mündlicher Noten}

{ \small
\setlength{\columnseprule}{0.4pt}
\setlength{\columnsep}{2cm}

\begin{multicols}{2}
\begin{itemize}[label={}]
\item \textbf{Sehr gut (15 -- 13 NP)}
\begin{itemize}
	\item Hervorragende Kenntnisse mit unterrichtstragendem Charakter
	\item Sprachlich korrekte, klare Ausdrucksweise, korrekte Fachsprache
	\item Selbständige Transferleistungen, Erkennen interdisziplinärer Zusammenhänge
\end{itemize}

\item \textbf{Gut (12 -- 10 NP)}
\begin{itemize}
	\item Gute Kenntnisse
	\item Sprachlich korrekte Ausdrucksweise, kleine fachlichsprachliche Unsicherheiten
	\item Transferleistungen sind (unter Anleitung) möglich
\end{itemize}

\item \textbf{Befriedigend (9 -- 7 NP)}
\begin{itemize}
	\item Deutlich erkennbares Bemühen um mündliche Teilnahme
	\item Befriedigende Kenntnisse (wechselnde Qualität, z.T. unklare Gedankengänge)
	\item Sprachliche und fachsprachliche Unklarheiten
	\item Transferleistungen sind unter Anleitung gelegentlich möglich
\end{itemize}

\item \textbf{Ausreichend (6 -- 4 NP)}
\begin{itemize}
	\item Ausreichende Kenntnisse (schwache Leistungen auf Befragung, häufiger unklare Gedankengänge)
	\item Unklare Ausdrucksweise, Unsicherheit in der Fachsprache
	\item Transferleistungen sind selbst unter Anleitung nur selten möglich
\end{itemize}

\item \textbf{Mangelhaft und schlechter (3 -- 0 NP)}
\begin{itemize}
	\item Häufig geistig abwesend
	\item Deutliche inhaltliche, sprachliche und fachsprachliche Mängel
	\item Keine Transferleistungen möglich
	\item Die Einziehung des Schülers / der Schülerin behindert häufig den Unterrichtsverlauf
\end{itemize}
\end{itemize}

\end{multicols}
}