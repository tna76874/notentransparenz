\section{Notenbildung (verbalisiert, vereinfachte Darstellung)}

\begin{enumerate}[label=\textbf{\arabic*)}]
	\item Schriftliche und mündliche Noten sind im Verhältnis $w_{sm} : 1$ gewichtet.
	\item Verbesserungen können in die schriftliche Note mit einfließen.
	\begin{enumerate}[label=\textbf{(\alph*)}]
		\item Verbesserungen werden immer im Bezug zu der aktuellen Note aus Kurztests und Klassenarbeiten gewichtet.
		\item Verbesserungen fließen nur in einem Bereich von $w_{th}$ um halbe Noten mit in die schriftliche Leistung ein. \textit{Beispiel}: $w_{th}=0{,}25$; Wenn der Schnitt aus Kurztest und Klassenarbeit zwischen z.B. $2{,}25$ und $2{,}75$ liegt, dann wird jede fertige Verbesserung mit $2{,}25$ gewertet und jede Fehlende mit $2{,}75$.
		\item Fehlerhafte oder unvollständige Verbesserungen verbessern oder verschlechtern den schriftlichen Schnitt nicht.
	\end{enumerate}
	\item Eine Anzahl von $n_{KT_0}$ Kurztests werden zusammen wie $w_{s_0}$ Klassenarbeiten gewertet. Unterschreitet die Anzahl der Kurztests $n_{KT_0}$, wird anteilig gewertet. Bei Überschreitung werden alle Kurztests zusammen wie  $w_{s_0}$ Klassenarbeiten gewertet. \textit{Beispiel}:  $n_{KT_0} = 3$; $w_{s_0}=1$; Werden nur zwei Kurztests geschrieben, dann werden diese zusammen soviel gewertet wie $\frac{2}{3}$ Klassenarbeiten. Werden fünf Kurztests geschrieben, werden diese zusammen soviel gewertet wie eine Klassenarbeit.
\end{enumerate}


