\documentclass{article}
\usepackage{ifthen}
\usepackage[a4paper, left=1.5cm, right=1.5cm, top=1.6cm, bottom=1.5cm]{geometry}
\usepackage{amsmath}
\usepackage{parskip}
\usepackage{enumitem}
\usepackage[german]{babel}
\usepackage{fancyhdr}
\usepackage{datetime}
\usepackage{hyperref}
\usepackage{tikz}
\hypersetup{
	colorlinks=true,
	linkcolor=black,
	urlcolor=black,
	citecolor=black
}
\usepackage{natbib}
\usepackage{multicol}
\usepackage{lastpage}

\usepackage{helvet}
\renewcommand{\familydefault}{\sfdefault}

\usepackage{titlesec}
\renewcommand{\thesection}{\Alph{section}}
\titleformat{\section}
{\normalsize\bfseries}{\thesection}{1em}{}
\titlespacing*{\section}{0pt}{0.5ex plus 0.2ex minus 0.1ex}{0.3ex plus 0.1ex}

%%%%%%%%%%%%%%% TOGGLE %%%%%%%%%%%%%%%%%%%%%%%%%%%%%%%%%%%%%
\newboolean{simple}
\setboolean{simple}{false}
%%%%%%%%%%%%%%%%%%%%%%%%%%%%%%%%%%%%%%%%%%%%%%%%%%%%%%%%%%%%

%%%%%%%%%%%%%%% IMPORTING PIPELINE TEX FILE IF PRESENT %%%%%
\IfFileExists{pipeline.tex}{\input{pipeline.tex}}{}
%%%%%%%%%%%%%%%%%%%%%%%%%%%%%%%%%%%%%%%%%%%%%%%%%%%%%%%%%%%%

% NEWCOMMANDS
\newcommand{\versiondir}{_version}
\newcommand{\standtex}{\input{\versiondir/timestamp_tex.tex}v\input{\versiondir/commit_count_tex.tex}}
\newcommand{\onlinedoc}{\href{https://transparenz.hilberg.eu}{https://transparenz.hilberg.eu}}

% NVO
\newcommand{\nvourl}{https://www.landesrecht-bw.de/bsbw/document/jlr-NotBildVBWrahmen}

% PARAMETERS
% Mittelwerte
\newcommand{\mkt}{\overline{m_{\mathrm{KT}}}}
\newcommand{\mfp}{\overline{m_{\mathrm{P}}}}
\newcommand{\mka}{\overline{m_{\mathrm{KA}}}}
\newcommand{\mm}{\overline{m_{\mathrm{m}}}}
\newcommand{\mseins}{\overline{m_{\mathrm{s}_1}}}
\newcommand{\ms}{\overline{m_{\mathrm{s}}}}
\newcommand{\mhalf}{m_{\mathrm{h}}}
% Anzahlen
\newcommand{\nkt}{n_{\mathrm{KT}}}
\newcommand{\nfp}{n_{\mathrm{P}}}
\newcommand{\nktnull}{n_{\mathrm{KT}_0}}
\newcommand{\nka}{n_{\mathrm{KA}}}
\newcommand{\nm}{n_{\mathrm{m}}}

\newcommand{\nvg}{n_{\mathrm{v}_\mathrm{g}}}
\newcommand{\nveins}{n_{\mathrm{v}_1}}
\newcommand{\nvzwei}{n_{\mathrm{v}_2}}
\newcommand{\nvo}{n_{\mathrm{v}_\mathrm{o}}}

% Gewichte
\newcommand{\wnull}{w_{0}}
\newcommand{\wsnull}{w_{\mathrm{s}_0}}
\newcommand{\ws}{w_{\mathrm{s}}}
\newcommand{\wdis}{w_{\mathrm{d}}}
\newcommand{\wth}{w_{\mathrm{th}}}
\newcommand{\wveins}{w_{\mathrm{v}_1}}
\newcommand{\wvzwei}{w_{\mathrm{v}_2}}
\newcommand{\wvdrei}{w_{\mathrm{v}_3}}
\newcommand{\wvvier}{w_{\mathrm{v}_4}}
\newcommand{\wsm}{w_{\mathrm{sm}}}

\fancypagestyle{pagestylesimple}{
	\fancyhf{}
	\fancyhead[L]{\large Notentransparenz}
	\fancyhead[C]{Seite \thepage\ von~\pageref{LastPage}}
	\fancyhead[R]{}
	\fancyfoot[L]{2024-07-20v2L
}
	\fancyfoot[C]{\footnotesize ID: \href{https://github.com/tna76874/notentransparenz/commit/a114884c171967087888f8dc694e1303b1491168}{a114884c171967087888f8dc694e1303b1491168}
 }
	\fancyfoot[R]{}
}

\fancypagestyle{pagestylefull}{
	\fancyhf{}
	\fancyhead[L]{\large Notentransparenz}
	\fancyhead[C]{Seite \thepage\ von~\pageref{LastPage}}
	\fancyhead[R]{Hilberg}
	\fancyfoot[L]{\href{https://transparenz.hilberg.eu/2024-08-01v1}{2024-08-01v1}
}
	\fancyfoot[C]{\footnotesize ID: \href{https://github.com/tna76874/notentransparenz/commit/2bc6c188deb960727ce6981a82612f767c5f5096}{2bc6c188deb960727ce6981a82612f767c5f5096}
 }
	\fancyfoot[R]{\onlinedoc}
}

\ifthenelse{\boolean{simple}}{
\pagestyle{pagestylesimple}
}{
\pagestyle{pagestylefull}
}

\begin{document}
	\vspace*{-1cm}
	In diesem Dokument zur Notentransparenz ist nach \href{\nvourl}{Notenbildungsverordnung (NVO) §7-§10} \cite{nvo} dargelegt, wie in der Regel die verschiedenen Leistungen bei der Notenbildung gewichtet werden.

	Die Notenbildung ist in Abschnitt C exakt definiert. Zur besseren Verständlichkeit wird diese in Abschnitt A vereinfacht und exemplarisch dargestellt. Die Reduktion in Abschnitt A hat keinen Anspruch auf Vollständigkeit.

	\ifthenelse{\boolean{simple}}{
	\begin{description}
		\item[Version Notentransparenz] 2024-07-20v2L
\\ID: \href{https://github.com/tna76874/notentransparenz/commit/a114884c171967087888f8dc694e1303b1491168}{a114884c171967087888f8dc694e1303b1491168}

	\end{description}
	}{
	\begin{description}
		\item[Version Notentransparenz] \href{https://transparenz.hilberg.eu/2024-08-01v1}{2024-08-01v1}
\\abrufbar unter \href{https://transparenz.hilberg.eu/2024-07-31v2}{https://transparenz.hilberg.eu/2024-07-31v2}
\\ ID: \href{https://github.com/tna76874/notentransparenz/commit/2bc6c188deb960727ce6981a82612f767c5f5096}{2bc6c188deb960727ce6981a82612f767c5f5096}

		\item[Version dieses Dokumentes] \href{https://transparenz.hilberg.eu/2024-07-23v3/files/tex/notentransparenz.pdf}{2024-07-21v2-DOC}
\\abrufbar unter \href{https://transparenz.hilberg.eu/2024-07-30v1/nt.pdf}{https://transparenz.hilberg.eu/2024-07-30v1/nt.pdf}
\\ ID: \href{https://github.com/tna76874/notentransparenz/commit/a114884c171967087888f8dc694e1303b1491168}{a114884c171967087888f8dc694e1303b1491168}

	\end{description}
	}
		
	\rule{\linewidth}{2pt}
	\renewcommand{\contentsname}{Abschnitte der Notentransparenz}
	\tableofcontents
	\rule{\linewidth}{2pt}
	
	\section{Notenbildung (verbalisiert, vereinfachte Darstellung)}
{\normalsize
\begin{enumerate}[label=\textbf{\arabic*)}]
	\item Schriftliche und mündliche Noten sind im Verhältnis $w_{sm} : 1$ gewichtet.
	\item Verbesserungen können in die schriftliche Note mit einfließen.
	\begin{enumerate}[label=\textbf{(\alph*)}]
		\item Verbesserungen werden immer im Bezug zu der aktuellen Note aus Kurztests und Klassenarbeiten gewichtet.
		\item Verbesserungen fließen nur in einem Bereich von $w_{th}$ um halbe Noten mit in die schriftliche Leistung ein. \textit{Beispiel}: $w_{th}=0{,}25$; Wenn der Schnitt aus Kurztest und Klassenarbeit zwischen z.B. $2{,}25$ und $2{,}75$ liegt, dann wird jede fertige Verbesserung mit $2{,}25$ gewertet und jede Fehlende mit $2{,}75$.
		\item Fehlerhafte oder unvollständige Verbesserungen verbessern oder verschlechtern den schriftlichen Schnitt nicht.
		\item Die Gewichtung der Verbesserungen entspricht, äquivalent umgerechnet auf die jeweiligen Randbereiche des Notensystems im Bezug zu dem relativen Anteil des Einflussbereichs $w_{th}$, zwei Klassenarbeiten.
	\end{enumerate}
	\item Eine Anzahl von $n_{KT_0}$ Kurztests werden zusammen wie $w_{s_0}$ Klassenarbeiten gewertet. Unterschreitet die Anzahl der Kurztests $n_{KT_0}$, wird anteilig gewertet. Bei Überschreitung werden alle Kurztests zusammen wie  $w_{s_0}$ Klassenarbeiten gewertet. \textit{Beispiel}:  $n_{KT_0} = 3$; $w_{s_0}=1$; Werden nur zwei Kurztests geschrieben, dann werden diese zusammen soviel gewertet wie $\frac{2}{3}$ Klassenarbeiten. Werden fünf Kurztests geschrieben, werden diese zusammen soviel gewertet wie eine Klassenarbeit.
\end{enumerate}
}


	
	\vspace{0.2cm}

	\section{Kriterien für die Erteilung mündlicher Noten}\label{chap:B}
{ 
\setlength{\columnseprule}{0.4pt}
\setlength{\columnsep}{2cm}

\begin{multicols}{2}
\begin{itemize}[label={}]
\item \textbf{Sehr gut (15 – 13 NP)}
\begin{itemize}
	\item Hervorragende Kenntnisse mit unterrichtstragendem Charakter
	\item Sprachlich korrekte, klare Ausdrucksweise, korrekte Fachsprache
	\item Selbständige Transferleistungen, Erkennen interdisziplinärer Zusammenhänge
\end{itemize}

\item \textbf{Gut (12 – 10 NP)}
\begin{itemize}
	\item Gute Kenntnisse
	\item Sprachlich korrekte Ausdrucksweise, kleine fachlichsprachliche Unsicherheiten
	\item Transferleistungen sind (unter Anleitung) möglich
\end{itemize}

\item \textbf{Befriedigend (9 – 7 NP)}
\begin{itemize}
	\item Deutlich erkennbares Bemühen um mündliche Teilnahme
	\item Befriedigende Kenntnisse (wechselnde Qualität, z.T. unklare Gedankengänge)
	\item Sprachliche und fachsprachliche Unklarheiten
	\item Transferleistungen sind unter Anleitung gelegentlich möglich
\end{itemize}

\item \textbf{Ausreichend (6 – 4 NP)}
\begin{itemize}
	\item Ausreichende Kenntnisse (schwache Leistungen auf Befragung, häufiger unklare Gedankengänge)
	\item Unklare Ausdrucksweise, Unsicherheit in der Fachsprache
	\item Transferleistungen sind selbst unter Anleitung nur selten möglich
\end{itemize}

\item \textbf{Mangelhaft und schlechter (3 – 0 NP)}
\begin{itemize}
	\item Häufig geistig abwesend
	\item Deutliche inhaltliche, sprachliche und fachsprachliche Mängel
	\item Keine Transferleistungen möglich
	\item Die Einziehung des Schülers / der Schülerin behindert häufig den Unterrichtsverlauf
\end{itemize}
\end{itemize}

\end{multicols}
}

	\clearpage
	\vspace*{-0.8cm}
\section{Notenbildung (exakte Definitionen)}

% DEFINITIONEN
\ifthenelse{\boolean{simple}}{
	\newcommand{\thisfontsize}{\scriptsize}
}{
	\newcommand{\thisfontsize}{\scriptsize}
}


\newcommand{\defformat}[1]{\tikz[baseline=(char.base)]{\node[draw,rectangle,inner sep=2pt] (char) {\textit{#1}};}}

\newcommand{\defka}{\defformat{KA}}
\newcommand{\defgfs}{\defformat{GFS}}
\newcommand{\defkt}{\defformat{KT}}
\newcommand{\defktp}{\defformat{P}}
\newcommand{\defsl}{\defformat{S}}
\newcommand{\defv}{\defformat{V}}
\newcommand{\defm}{\defformat{M}}
\newcommand{\defe}{\defformat{E}}
\newcommand{\defsn}{\defformat{N}}
\newcommand{\defsnp}{\defformat{NP}}

\newcommand{\catka}{\defka~und~\defgfs}
\newcommand{\catkt}{\defkt~und~\defsl~und~\defktp}
\newcommand{\catm}{\defm~und~\defe}

\begin{enumerate}[label=\textbf{(\arabic*)}, align=left, leftmargin=*]
	\item \textbf{Definitionen} { \thisfontsize \mbox{\textbf{\defka}: Klassenarbeit}; \mbox{\textbf{\defgfs}: gleichwertige Feststellung von Leistungen der Schüler}; \mbox{\textbf{\defkt}: schriftliche Wiederholungsarbeit};  \linebreak \mbox{\textbf{\defktp}: praktische Leistung gewichtet wie ein \defkt}; \mbox{\textbf{\defsl}: sonstige schriftliche Leistung gewichtet wie ein \defkt}; \ifthenelse{\boolean{simple}}{}{\mbox{\textbf{\defv}: Verbesserung (schriftliche Leistung)};}  \linebreak \mbox{\textbf{\defm}: mündliche Leistung}; \mbox{\textbf{\defe}: gesondert bewertete mündliche Leistung}; \textit{Notensysteme}: Noten (\textbf{\defsn}), Notenpunkte (\textbf{\defsnp}) }
	
	\item \textbf{Arithmetische Mittelwerte der Noten}  { \thisfontsize $\mkt$ (\catkt); $\mka$ (\catka); $\mm$ (\catm) }
	
	\ifthenelse{\boolean{simple}}{}{
	\item \textbf{Korrekturzeichen} { \thisfontsize  Es werden die Korrekturzeichen der jeweils aktuellen \textit{\glqq Beurteilungs- und Korrekturrichtlinien für die Abiturprüfung\grqq{}} übernommen. }
	}
	
	\item \textbf{Ankündigung von Leistungen} { \thisfontsize \defkt, Nachschreibearbeiten (\defkt und \defka), \defsl, \defktp, \catm~sind in der Regel unangekündigt.}
	\item \textbf{Anzahl der Leistungen} { \thisfontsize $\nkt$ (\catkt); $\nka$ (\catka); $\nm$ (\catm) \href{\nvourl}{nach NVO} \cite{nvo}}.
	
	\ifthenelse{\boolean{simple}}{}{
	\item \textbf{Verbesserungen} { \thisfontsize Definition von \defv: \glqq\textit{Zu den Leistungen mit Verbesserungsstatus ungleich \glqq ---\grqq{} müssen \textit{individuelle} Verbesserungen angefertigt werden. Diese beinhalten verbalisierte Fehleranalysen jedes mit einem Korrekturzeichen markierten \textit{persönlichen} Fehlers und dessen Korrektur. Eine Verbesserung ist explizit nicht, die Definition des entsprechenden Korrekturzeichens zu nennen oder zu beschreiben. Zusätzlich zu der Fehleranalyse muss bei mehreren aufeinanderfolgenden Fehlern die fehlerhafte Aufgabe noch einmal vollständig richtig notiert werden.}\grqq{} Falls bei einer Abgabe kein Inhalt nach der Definition einer Verbesserung erkennbar ist, dann wird der jeweilige Verbesserungsstatus als \glqq fehlt\grqq{} gesetzt.  Falls durch mündliche Nachfrage festgestellt werden kann, dass die Verbesserung nicht selbstständig angefertigt und verstanden wurde, dann wird der Verbesserungsstatus entsprechend als \glqq fehlt\grqq{} gesetzt bzw. geändert. Die maßgebliche Berechnung des Notenanteils der Verbesserungen erfolgt zum Tag der jeweiligen Zeugniskonferenz datiert auf den Zeitpunkt der jeweiligen Leistung.}
	\item \textbf{Abgabe der Verbesserungen} { \thisfontsize Eine \glqq \textit{Abgabe}\grqq{} ist die Dokumentation des Standes der persönlichen Verbesserung eines Schülers / einer Schülerin. Für diese ist der Schüler / die Schülerin selber verantwortlich. Die Dokumentation kann digital als Scan/Foto des Originals über einen zugestellten Link erfolgen oder analog, indem von der Lehrkraft vor Ort die Verbesserung datiert wird. Da die Verbesserung Teil der persönlichen Lernunterlagen ist, verbleibt diese bei dem Schüler / der Schülerin und muss auf Nachfrage der Lehrkraft im Original, inklusive der Original-Dokumentation (\textit{digital}: Link zu hochgeladener Datei; \textit{analog}: von der Lehrkraft datiertes Papier) der Datierung, vorgelegt werden können. Die Abgabe erfolgt innerhalb einer einwöchigen Frist nach der Rückgabe der Arbeit. Abgaben auf anderem digitalen Weg (z.B. Email, Messenger, ... ) sind ausgeschlossen, da dort die Übertragung nicht gewährleistet ist. Falls die Vorlage der Original-Verbesserung inklusive Original-Dokumentation/Datierung nicht möglich ist, wird der jeweilige Verbesserungsstatus auf \glqq \textit{fehlt}\grqq{} gesetzt bzw. geändert.} 
	
	\item \textbf{Anzahl von zu verbessernden Leistungen} { \thisfontsize \mbox{} \
	\setlength{\columnsep}{-20pt}
	\begin{multicols}{2}
	\begin{enumerate}[label=\textbf{\textbullet}, align=left, leftmargin=*]
		\item[\textit{Gesamtanzahl}] $\nvg$ (Verbesserungsstatus ist nicht \glqq ---\grqq{})
		\item[\textit{fehlend}] $\nveins$ (Verbesserungsstatus ist \glqq \textit{fehlt}\grqq{})
		\item[\textit{fertig}] $\nvzwei$ (Verbesserungsstatus ist \glqq \textit{fertig}\grqq{})
		\item[\textit{nicht abgeschlossenen}] $\nvo$ (Verbesserungsstatus ist nicht: \glqq\textit{---}\grqq{}, \glqq\textit{fehlt}\grqq{}, \glqq\textit{fertig}\grqq{})
	\end{enumerate}
	\end{multicols}
	}
	\vspace*{-12pt}
	}
	
	\item \textbf{Gewichtungsfaktor $\mka$ und $\mkt$}: $\wsnull$; $\nktnull$  (Falls nicht anders mitgeteilt, ist $\wsnull=1$ und $\nktnull=3$)
	\[
	\ws =
	\begin{cases}
		\frac{\nkt \cdot \wsnull}{\nktnull} & \text{für }    \nkt < \nktnull \\
		\wsnull & \text{für }    \nkt \geq \nktnull \\
	\end{cases}
	\]
	\ifthenelse{\boolean{simple}}{}{	
	\item \textbf{Schnitt $\mka$ und $\mkt$} \vspace{-0.5cm}
	\[
	\mseins = \frac{ \nka \cdot \mka + \ws \cdot \mkt }{\nka + \ws}
	\]
	\item \textbf{Diskretisierung} $\wdis$; $\wth$; $\mhalf$; $\wnull$ (Falls nicht anders mitgeteilt, ist $\wth=0{,}25$) \cite{wikigaussklammer,wikibetrag,wikimodulo}
	\[
	\wdis = 
	\begin{cases}
		1 & \text{für }  \wth = 0 \lor \left | \wth \right | > 0{,}5 \lor \nveins = \nvzwei\\
		\left| \frac{0{,}5 - (\mseins \mod 1)}{\wth} \right| & \text{ansonsten}\\
	\end{cases}
	\,\hspace{0.1cm}\,
	\mhalf = \frac{\lceil \mseins \rceil + \lfloor \mseins \rfloor}{2}
	\,\hspace{0.1cm}\,
	\wnull =
	\begin{cases}
		6-1 & \text{ für \defsn} \\
		15 - 0 & \text{ für \defsnp} \\
	\end{cases}
	\]
	\item \textbf{Gewichtungsfaktoren Verbesserung} $\wveins$; $\wvzwei$; $\wvdrei$; $\wvvier$ \cite{wikilog}
	\[
	\wveins =
	\begin{cases}
		0 & \text{für }  \wdis \geq 1 \\
		\mhalf + \left | \wth \right | & \text{für }    \wdis < 1 \text{ und \defsn} \\
		\mhalf - \left | \wth \right |& \text{für }    \wdis < 1 \text{ und \defsnp} \\
	\end{cases}
	\,\,\,\,
	\wvzwei =
	\begin{cases}
		0 & \text{für }  \wdis \geq 1 \\
		\mhalf - \left | \wth \right | & \text{für }    \wdis < 1 \text{ und \defsn} \\
		\mhalf + \left | \wth \right | & \text{für }    \wdis < 1 \text{ und \defsnp} \\
	\end{cases}
	\]
	\[
	\wvdrei =
	\begin{cases}
		0 & \text{für } \wdis \geq 1\\
		\left | \frac{\wnull}{\wth}  \right | & \text{ansonsten}\\
	\end{cases}
	\,\hspace{0.5cm}\,
	\wvvier =
	\begin{cases}
		0 & \text{für }  \wdis \geq 1 \\
		\frac{\nveins \cdot \wveins + \nvo \cdot \mseins + \nvzwei \cdot \wvzwei}{\nvg} & \text{ansonsten}\\
	\end{cases}
	\]
	}
	

	\item \textbf{Schnitt schriftlich}
	\ifthenelse{\boolean{simple}}{
	\[
	\ms = \frac{ \nka \cdot \mka + \ws \cdot \mkt  }{\nka + \ws}
	\]
	}{
	\vspace{-0.3cm}
	\[
	\ms = \frac{ \nka \cdot \mka + \ws \cdot \mkt +\wvdrei \cdot \wvvier   }{\nka + \ws + \wvdrei}
	\]
	}
	\item \textbf{Gewichtungsfaktor schriftlich/mündlich} $\wsm$ (Falls nicht anders mitgeteilt, ist $\wsm=3$)
	\item \textbf{Gesamtnote $GN$}
	\[
	GN = \frac{\wsm \cdot \ms + \mm}{\wsm+1}
	\]
\end{enumerate}

\renewcommand\refname{\footnotesize Quellen}
\renewcommand{\bibname}{Quellen}

\vfill


\begingroup
\tiny
\setlength{\bibsep}{0pt}
\bibliographystyle{plain}
\bibliography{references}
\endgroup

\ifthenelse{\boolean{simple}}{}{
\vspace{-0.2cm}
{\tiny Die Onlineversion \onlinedoc \, ist ebenfalls Teil der Notentransparenz und gültig zum angegebenen Stand. Alle Änderungen sind datiert und können online auch im Nachhinein abgerufen werden.}
}
\end{document}
