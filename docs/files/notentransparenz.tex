\documentclass{article}
\usepackage[a4paper, margin=1.8cm]{geometry}
\usepackage{amsmath}
\usepackage{parskip}
\usepackage{enumitem}
\usepackage[german]{babel}
\usepackage{fancyhdr}
\usepackage{datetime}
\usepackage{hyperref}
\usepackage{catchfilebetweentags}

\usepackage{helvet}
\renewcommand{\familydefault}{\sfdefault}

\newcommand{\onlinedoc}{\href{https://transparenz.hilberg.eu}{transparenz.hilberg.eu}}

\fancypagestyle{meinpagestyle}{
	\fancyhf{}
	\fancyhead[L]{\large Notentransparenz}
	\fancyhead[R]{Hilberg}
	\fancyfoot[L]{Stand 
 }
	\fancyfoot[C]{\footnotesize \href{https://github.com/tna76874/notentransparenz/tree/}{Version }
 }
	\fancyfoot[R]{\onlinedoc}
}
\pagestyle{meinpagestyle}

\begin{document}

\begin{enumerate}[label=\textbf{\textbullet}, align=left, leftmargin=*]
	\item[\textbf{Definitionen}] \textbf{KA}: Klassenarbeit; \textbf{GFS}: GFS; \textbf{KT}: schriftliche Wiederholungsarbeit oder praktische Arbeit; \textbf{m}: mündlich; Notensysteme: Noten (\textbf{N}), Notenpunkte (\textbf{NP})
	\item[\textbf{Mittelwerte der Noten}] $\overline{m_{KT}}$ (KT); $\overline{m_{KA}}$ (KA und GFS); $\overline{m_{m}}$ (m)
	\item[\textbf{Anzahl der Leistungen}] $n_{KT}$; $n_{KA}$; $n_{m}$
	\item[\textbf{Verbesserungen}] Zu den Leistungen mit Verbesserungsstatus ungleich \glqq ---\grqq{} müssen \textit{individuelle} Verbesserungen angefertigt werden. Diese beinhalten Fehleranalysen jedes mit einem Korrekturzeichen markierten \textit{persönlichen} Fehlers und dessen Korrektur.
	\item[\textbf{Abgabe der Verbesserungen}] Eine \glqq \textit{Abgabe}\grqq{} ist die Dokumentation des Standes der persönlichen Verbesserung eines Schülers / einer Schülerin. Für diese ist der Schüler / die Schülerin selber verantwortlich. Die Dokumentation kann digital über einen zugestellten Link erfolgen oder analog, indem von der Lehrkraft vor Ort die Verbesserung datiert und fotografiert wird. Da die Verbesserung Teil der persönlichen Lernunterlagen ist, verbleibt diese bei dem Schüler / der Schülerin und muss auf Nachfrage der Lehrkraft im Original vorgelegt werden können. Die Abgabe erfolgt innerhalb einer Wochenfrist nach der Rückgabe der Arbeit.
	\item[\textbf{Anzahl von zu verbessernden Leisungen}] \mbox{} \
	\begin{enumerate}[label=\textbf{\textbullet}, align=left, leftmargin=*]
		\item[\textit{Gesamtanzahl}] $n_{v_g}$ (Verbesserungsstatus ist nicht \glqq ---\grqq{})
		\item[\textit{fehlend}] $n_{v_1}$ (Verbesserungsstatus ist \glqq \textit{fehlt}\grqq{})
		\item[\textit{fertig}] $n_{v_2}$ (Verbesserungsstatus ist \glqq \textit{fertig}\grqq{})
		\item[\textit{nicht abgeschlossenen}] $n_{v_o}$ (Verbesserungsstatus ist nicht: \glqq\textit{---}\grqq{}, \glqq\textit{fehlt}\grqq{}, \glqq\textit{fertig}\grqq{})
	\end{enumerate}
	\item[\textbf{Gewichtungsfaktor KA/KT}] $w_{s_0}$ (Falls nicht anders mitgeteilt, ist $w_{s_0}=1$)
	\[
	w_s =
	\begin{cases}
		0 & \text{für }  n_{KT} = 0 \\
		\frac{w_{s_0}}{2} & \text{für }    n_{KT} = 1 \\
		w_{s_0} & \text{für }    n_{KT} > 1 \\
	\end{cases}
	\]	
	\item[\textbf{Mittelwert KA und KT}] 
	\[
	\overline{m_{s_1}} = \frac{ n_{KA} \cdot \overline{m_{KA}} + w_s \cdot \overline{m_{KT}} }{n_{KA} + w_s}
	\]
	\item[\textbf{Diskretisierungsfaktor}] $w_{d}$ (Falls nicht anders mitgeteilt, ist $w_{th}=0{,}25$) \cite{wikimodulo,wikibetrag}
	\[
	w_{d} = \left| \frac{0{,}5 - (\overline{m_{s_1}} \mod 1)}{w_{th}} \right|
	\]
	\item[\textbf{Gewichtungsfaktoren Verbesserung}] $w_{v_1}$; $w_{v_2}$; $w_{v_3}$;$w_{v_4}$ \cite{wikigaussklammer} \\
	\[
	w_{v_1} =
	\begin{cases}
		0 & \text{für }  w_{d} > 1 \\
		\lceil \overline{m_{s_1}} \rceil & \text{für }    w_{d} \leq 1 \text{ und Notensystem N} \\
		\lfloor \overline{m_{s_1}} \rfloor & \text{für }    w_{d} \leq 1 \text{ und Notensystem NP} \\
	\end{cases}
	\,\,\,\,
	w_{v_2} =
	\begin{cases}
		0 & \text{für }  w_{d} > 1 \\
		\lfloor \overline{m_{s_1}} \rfloor & \text{für }    w_{d} \leq 1 \text{ und Notensystem N} \\
		\lceil \overline{m_{s_1}} \rceil & \text{für }    w_{d} \leq 1 \text{ und Notensystem NP} \\
	\end{cases}
	\]
	
	\[
	w_{v_3} =
	\begin{cases}
		0 & \text{für }  w_{d} \geq 1 \text{ oder } n_{v_g}=0 \\
		10 & \text{für }    w_{d} < 1 \\
	\end{cases}
	\,\,\,\,\,\,\,\,
	w_{v_4} =
	\begin{cases}
		0 & \text{für }  w_{d} \geq 1 \text{ oder } n_{v_g}=0 \\
		\frac{n_{v_1} \cdot w_{v_1} + n_{v_o} \cdot \overline{m_{s_1}} + w_{v_2} + n_{v_2} \cdot w_{v_2}}{n_{v_g}} & \text{für }    w_{d} < 1 \\
	\end{cases}
	\]	
	\item[\textbf{Mittelwert schriftlich}]
	\[
	\overline{m_{s}} = \frac{ n_{KA} \cdot \overline{m_{KA}} + w_s \cdot \overline{m_{KT}} +w_{v_3} \cdot w_{v_4}   }{n_{KA} + w_s + w_{v_3}}
	\]
	\item[\textbf{Gewichtungsfaktor schriftlich/mündlich}] $w_{sm}$ (Falls nicht anders mitgeteilt, ist $w_{sm}=3$)
	\item[\textbf{Gesamtnote $GN$}]
	\[
	GN = \frac{w_{sm} \cdot \overline{m_{s}} + \overline{m_{m}}}{w_{sm}+1}
	\]
\end{enumerate}

\renewcommand\refname{\footnotesize Mathematische Notationen}
\renewcommand{\bibname}{Quellen}

\vfill

\begingroup
\tiny
\bibliographystyle{plain}
\bibliography{references}
\endgroup

{\tiny Die Onlineversion \onlinedoc \, ist ebenfalls Teil der Notentransparenz und gültig zum Ausgabezeitpunkt dieses Dokumentes. Alle Änderungen sind datiert und können online auch im Nachhinein abgerufen werden.}

\end{document}
