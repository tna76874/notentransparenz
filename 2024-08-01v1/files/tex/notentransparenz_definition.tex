\vspace*{-0.8cm}
\section{Notenbildung (exakte Definitionen)}

% DEFINITIONEN
\ifthenelse{\boolean{simple}}{
	\newcommand{\thisfontsize}{\scriptsize}
}{
	\newcommand{\thisfontsize}{\scriptsize}
}


\newcommand{\defformat}[1]{\tikz[baseline=(char.base)]{\node[draw,rectangle,inner sep=2pt] (char) {\textit{#1}};}}

\newcommand{\defka}{\defformat{KA}}
\newcommand{\defgfs}{\defformat{GFS}}
\newcommand{\defkt}{\defformat{KT}}
\newcommand{\defktp}{\defformat{P}}
\newcommand{\defsl}{\defformat{S}}
\newcommand{\defv}{\defformat{V}}
\newcommand{\defm}{\defformat{M}}
\newcommand{\defe}{\defformat{E}}
\newcommand{\defsn}{\defformat{N}}
\newcommand{\defsnp}{\defformat{NP}}

\newcommand{\catka}{\defka~und~\defgfs}
\newcommand{\catkt}{\defkt~und~\defsl~und~\defktp}
\newcommand{\catm}{\defm~und~\defe}

\begin{enumerate}[label=\textbf{(\arabic*)}, align=left, leftmargin=*]
	\item \textbf{Definitionen} { \thisfontsize \mbox{\textbf{\defka}: Klassenarbeit}; \mbox{\textbf{\defgfs}: gleichwertige Feststellung von Leistungen der Schüler}; \mbox{\textbf{\defkt}: schriftliche Wiederholungsarbeit};  \linebreak \mbox{\textbf{\defktp}: praktische Leistung gewichtet wie ein \defkt}; \mbox{\textbf{\defsl}: sonstige schriftliche Leistung gewichtet wie ein \defkt}; \ifthenelse{\boolean{simple}}{}{\mbox{\textbf{\defv}: Verbesserung (schriftliche Leistung)};}  \linebreak \mbox{\textbf{\defm}: mündliche Leistung}; \mbox{\textbf{\defe}: gesondert bewertete mündliche Leistung}; \textit{Notensysteme}: Noten (\textbf{\defsn}), Notenpunkte (\textbf{\defsnp}) }
	
	\item \textbf{Arithmetische Mittelwerte der Noten}  { \thisfontsize $\mkt$ (\catkt); $\mka$ (\catka); $\mm$ (\catm) }
	
	\ifthenelse{\boolean{simple}}{}{
	\item \textbf{Korrekturzeichen} { \thisfontsize  Es werden die Korrekturzeichen der jeweils aktuellen \textit{\glqq Beurteilungs- und Korrekturrichtlinien für die Abiturprüfung\grqq{}} übernommen. }
	}
	
	\item \textbf{Ankündigung von Leistungen} { \thisfontsize \defkt, Nachschreibearbeiten (\defkt und \defka), \defsl, \defktp, \catm~sind in der Regel unangekündigt.}
	\item \textbf{Anzahl der Leistungen} { \thisfontsize $\nkt$ (\catkt); $\nka$ (\catka); $\nm$ (\catm) \href{\nvourl}{nach NVO} \cite{nvo}}.
	
	\ifthenelse{\boolean{simple}}{}{
	\item \textbf{Verbesserungen} { \thisfontsize Definition von \defv: \glqq\textit{Zu den Leistungen mit Verbesserungsstatus ungleich \glqq ---\grqq{} müssen \textit{individuelle} Verbesserungen angefertigt werden. Diese beinhalten verbalisierte Fehleranalysen jedes mit einem Korrekturzeichen markierten \textit{persönlichen} Fehlers und dessen Korrektur. Eine Verbesserung ist explizit nicht, die Definition des entsprechenden Korrekturzeichens zu nennen oder zu beschreiben. Zusätzlich zu der Fehleranalyse muss bei mehreren aufeinanderfolgenden Fehlern die fehlerhafte Aufgabe noch einmal vollständig richtig notiert werden.}\grqq{} Falls bei einer Abgabe kein Inhalt nach der Definition einer Verbesserung erkennbar ist, dann wird der jeweilige Verbesserungsstatus als \glqq fehlt\grqq{} gesetzt.  Falls durch mündliche Nachfrage festgestellt werden kann, dass die Verbesserung nicht selbstständig angefertigt und verstanden wurde, dann wird der Verbesserungsstatus entsprechend als \glqq fehlt\grqq{} gesetzt bzw. geändert. Die maßgebliche Berechnung des Notenanteils der Verbesserungen erfolgt zum Tag der jeweiligen Zeugniskonferenz datiert auf den Zeitpunkt der jeweiligen Leistung.}
	\item \textbf{Abgabe der Verbesserungen} { \thisfontsize Eine \glqq \textit{Abgabe}\grqq{} ist die Dokumentation des Standes der persönlichen Verbesserung eines Schülers / einer Schülerin. Für diese ist der Schüler / die Schülerin selber verantwortlich. Die Dokumentation kann digital als Scan/Foto des Originals über einen zugestellten Link erfolgen oder analog, indem von der Lehrkraft vor Ort die Verbesserung datiert wird. Da die Verbesserung Teil der persönlichen Lernunterlagen ist, verbleibt diese bei dem Schüler / der Schülerin und muss auf Nachfrage der Lehrkraft im Original, inklusive der Original-Dokumentation (\textit{digital}: die Originaldatei, die digital abgegeben wurde mit SHA-256-Prüfsumme z.B. \cite{emn178SHA256File}; \textit{analog}: von der Lehrkraft datiertes Papier) der Datierung, vorgelegt werden können. Die Abgabe erfolgt innerhalb einer einwöchigen Frist nach der Rückgabe der Arbeit. Abgaben auf anderem digitalen Weg (z.B. Email, Messenger, ... ) sind ausgeschlossen, da dort die Übertragung nicht gewährleistet ist. Falls die Vorlage der Original-Verbesserung inklusive Original-Dokumentation/Datierung nicht möglich ist, wird der jeweilige Verbesserungsstatus auf \glqq \textit{fehlt}\grqq{} gesetzt bzw. geändert.} 
	
	\item \textbf{Anzahl von zu verbessernden Leistungen} { \thisfontsize \mbox{} \
	\setlength{\columnsep}{-20pt}
	\begin{multicols}{2}
	\begin{enumerate}[label=\textbf{\textbullet}, align=left, leftmargin=*]
		\item[\textit{Gesamtanzahl}] $\nvg$ (Verbesserungsstatus ist nicht \glqq ---\grqq{})
		\item[\textit{fehlend}] $\nveins$ (Verbesserungsstatus ist \glqq \textit{fehlt}\grqq{})
		\item[\textit{fertig}] $\nvzwei$ (Verbesserungsstatus ist \glqq \textit{fertig}\grqq{})
		\item[\textit{nicht abgeschlossenen}] $\nvo$ (Verbesserungsstatus ist nicht: \glqq\textit{---}\grqq{}, \glqq\textit{fehlt}\grqq{}, \glqq\textit{fertig}\grqq{})
	\end{enumerate}
	\end{multicols}
	}
	\vspace*{-12pt}
	}
	
	\item \textbf{Gewichtungsfaktor $\mka$ und $\mkt$}: $\wsnull$; $\nktnull$  (Falls nicht anders mitgeteilt, ist $\wsnull=1$ und $\nktnull=3$)
	\[
	\ws =
	\begin{cases}
		\frac{\nkt \cdot \wsnull}{\nktnull} & \text{für }    \nkt < \nktnull \\
		\wsnull & \text{für }    \nkt \geq \nktnull \\
	\end{cases}
	\]
	\ifthenelse{\boolean{simple}}{}{	
	\item \textbf{Schnitt $\mka$ und $\mkt$} \vspace{-0.5cm}
	\[
	\mseins = \frac{ \nka \cdot \mka + \ws \cdot \mkt }{\nka + \ws}
	\]
	\item \textbf{Diskretisierung} $\wdis$; $\wth$; $\mhalf$; $\wnull$ (Falls nicht anders mitgeteilt, ist $\wth=0{,}25$) \cite{wikigaussklammer,wikibetrag,wikimodulo}
	\[
	\wdis = 
	\begin{cases}
		1 & \text{für }  \wth = 0 \lor \left | \wth \right | > 0{,}5 \lor \nveins = \nvzwei\\
		\left| \frac{0{,}5 - (\mseins \mod 1)}{\wth} \right| & \text{ansonsten}\\
	\end{cases}
	\,\hspace{0.1cm}\,
	\mhalf = \frac{\lceil \mseins \rceil + \lfloor \mseins \rfloor}{2}
	\,\hspace{0.1cm}\,
	\wnull =
	\begin{cases}
		6-1 & \text{ für \defsn} \\
		15 - 0 & \text{ für \defsnp} \\
	\end{cases}
	\]
	\vspace{-0.2cm}
	\item \textbf{Gewichtungsfaktoren Verbesserung} $\wveins$; $\wvzwei$; $\wvdrei$; $\wvvier$ \cite{wikilog}
	\vspace{-0.2cm}
	\[
	\wveins =
	\begin{cases}
		0 & \text{für }  \wdis \geq 1 \\
		\mhalf + \left | \wth \right | & \text{für }    \wdis < 1 \text{ und \defsn} \\
		\mhalf - \left | \wth \right |& \text{für }    \wdis < 1 \text{ und \defsnp} \\
	\end{cases}
	\,\,\,\,
	\wvzwei =
	\begin{cases}
		0 & \text{für }  \wdis \geq 1 \\
		\mhalf - \left | \wth \right | & \text{für }    \wdis < 1 \text{ und \defsn} \\
		\mhalf + \left | \wth \right | & \text{für }    \wdis < 1 \text{ und \defsnp} \\
	\end{cases}
	\]
	\vspace{-0.2cm}
	\[
	\wvdrei =
	\begin{cases}
		0 & \text{für } \wdis \geq 1\\
		\left | \frac{\wnull}{\wth}  \right | & \text{ansonsten}\\
	\end{cases}
	\,\hspace{0.5cm}\,
	\wvvier =
	\begin{cases}
		0 & \text{für }  \wdis \geq 1 \\
		\frac{\nveins \cdot \wveins + \nvo \cdot \mseins + \nvzwei \cdot \wvzwei}{\nvg} & \text{ansonsten}\\
	\end{cases}
	\]
	}
	\item \textbf{Schnitt schriftlich}
	\ifthenelse{\boolean{simple}}{
	\[
	\ms = \frac{ \nka \cdot \mka + \ws \cdot \mkt  }{\nka + \ws}
	\]
	}{
	\vspace{-0.5cm}
	\[
	\ms = \frac{ \nka \cdot \mka + \ws \cdot \mkt +\wvdrei \cdot \wvvier   }{\nka + \ws + \wvdrei}
	\]
	}
	\item \textbf{Gewichtungsfaktor schriftlich/mündlich} $\wsm$ (Falls nicht anders mitgeteilt, ist $\wsm=3$)
	\item \textbf{Gesamtnote $GN$}
	\[
	GN = \frac{\wsm \cdot \ms + \mm}{\wsm+1}
	\]
\end{enumerate}

\renewcommand\refname{\footnotesize Quellen}
\renewcommand{\bibname}{Quellen}

\vfill


\begingroup
\tiny
\setlength{\bibsep}{0pt}
\bibliographystyle{plain}
\bibliography{references}
\endgroup
\ifthenelse{\boolean{simple}}{}{
\vspace{-0.2cm}
{\tiny Die Onlineversion \onlinedoc \, ist ebenfalls Teil der Notentransparenz und gültig zum angegebenen Stand. Alle Änderungen sind datiert und können online auch im Nachhinein abgerufen werden.}
}